\documentclass{article}


%% For Table of Contents hyperlinks
\usepackage{color}   %May be necessary if you want to color links
\usepackage{hyperref}
\hypersetup{
    colorlinks=true, %set true if you want colored links
    linktoc=all,     %set to all if you want both sections and subsections linked
    linkcolor=black,  %choose some color if you want links to stand out
}
\usepackage{luacode}


\title{FRIL Help Documentation Rewritten}


\begin{document}
\maketitle


\begin{luacode*}
for line in io.lines("fril_help/FrilHelp.frh") do

	if string.len(line) > 1 then
		line = string.gsub(line, "_", "\\_")
	
		if string.sub(line, 1, 1) == "^" then 
			line = string.sub(line, 2, string.len(line))
			tex.print(string.format("\\section{%s}", line));
			seclabel = string.gsub(line, "\\_", "_")
			tex.print(string.format("\\label{sec:%s}", seclabel));
		else
			-- Bold any strings of the format "<text>"
			line = string.gsub(line, "<(.-)>", "<\\textbf{%1}>")
			
			--Add links to sections
			sec = string.match(line, "^(.-)/")
			if sec ~= nil then
				sec = string.gsub(sec, "\\_", "_")
				
				ref = string.format("\\hyperref[sec:%s]{\\textbf{%%1}}/", sec)
				line = string.gsub(line, "^(.-)/", ref)
			end
		
			tex.print(line)
			tex.print("\\\\")
		end
	end
end

\end{luacode*}



\end{document}